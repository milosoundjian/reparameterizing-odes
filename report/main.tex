\documentclass[oneside, a4paper, onecolumn, 11pt]{article}

% Change this: Customize the title, author, advisor, abstract
\newcommand{\thesistitle}[0]{Reparametrizing ODE models by scaling}
\newcommand{\authorname}[0]{Milos Oundjian}

\newcommand{\supervisor}[0]{Gleb Pogudin}
\newcommand{\supervisorinstitution}[0]{LIX}

\newcommand{\abstracttext}[0]{
Lorem ipsum dolor sit amet, consectetur adipiscing elit, sed do eiusmod tempor incididunt ut labore et dolore magna aliqua. Ut enim ad minim veniam, quis nostrud exercitation ullamco laboris nisi ut aliquip ex ea commodo consequat. Duis aute irure dolor in reprehenderit in voluptate velit esse cillum dolore eu fugiat nulla pariatur. Excepteur sint occaecat cupidatat non proident, sunt in culpa qui officia deserunt mollit anim id est laborum.
}

\usepackage[
    left=2cm,top=2.0cm,bottom=2.0cm,right=2cm,
    headheight=17pt, % as per the warning by fancyhdr
    includehead,includefoot,
    heightrounded, % to avoid spurious underfull messages
]{geometry}

% Added package
\usepackage[utf8]{inputenc}

\usepackage[T1]{fontenc}
\usepackage{amstext}
\usepackage{amsmath}
\usepackage{amssymb}
\usepackage{url}
\usepackage{graphicx}
\usepackage{wrapfig}
\usepackage{enumerate}
\usepackage{paralist}
\usepackage{xspace}
\usepackage{color}
\usepackage{times}
\usepackage[colorlinks, linkcolor=blue]{hyperref}
\setlength {\marginparwidth }{2cm}
\usepackage{pdfpages}
\usepackage{fancyhdr} %% For changing headers and footers

\usepackage{titling}
\usepackage[nottoc, numbib]{tocbibind}

% Added packages
\usepackage[framemethod=tikz]{mdframed} % For better box in abstract
\usepackage{esdiff} % For differentials
\usepackage{enumitem} % For step by step methodology

%% \predate{}
%% \postdate{}
%% \date{}
%% \author{\authorname}

\begin{document}

%\title{\thesistitle}

%\maketitle

% Max 10 lines.
%\noindent \paragraph*{Abstract}
%\abstract

\hspace{0pt}
\vfill

\begin{center}

    \includegraphics[width=0.3\textwidth]{logo-EP-vertical}

    \vspace*{2em}
    %
    {\large
        \textbf{\'Ecole Polytechnique}

        \vspace*{1em}
        \textit{BACHELOR THESIS IN COMPUTER SCIENCE}


        \vspace*{3em}
        {\Huge \textbf{\thesistitle}}
        \vspace*{3em}



        \textit{Author:}

        \vspace*{1em}
        \authorname{}, \'Ecole Polytechnique

        \vspace*{2em}
        %
        {\textit{Advisor:}}

        \vspace*{1em}
        \supervisor{}, \supervisorinstitution{}
    }

    \vspace*{2em}
    \textit{Academic year 2023/2024}

\end{center}

\vfill
\hspace{0pt}

\newpage

\noindent\textbf{Abstract}

\begin{mdframed}
    \abstracttext{}
\end{mdframed}

\newpage

% Setting up the header
\pagestyle{fancy}
%\renewcommand{\headrulewidth}{0pt} % Remove line at top
%\renewcommand{\headrulewidth}{0.4pt}% Default \headrulewidth is 0.4pt
\lhead{\authorname}
%\chead{\acronym}
\rhead{\thesistitle}

\newpage
\tableofcontents
\newpage

% \pagenumbering{arabic}

\section{Introduction}

The main overarching goal of this thesis is to implement an algorithm using scaling invariants and symmetry reduction of dynamical systems to improve the efficiency of solving ODEs computationally. What this means is that given an ODE, say for example the predator-prey model given by the following set of equations:
\begin{align*}
    \diff{n}{t}
     & = n (r (1 - \frac{n}{K} - k \frac{p}{n + d})), \\
    \diff{p}{t}
     & = sp (1 - h \frac{p}{n})
\end{align*}

In order to do this, I implemented in Julia, the necessary functions, such as getting the HNF multipliers in both row and column form. Most of the references for this. These include the functions \cite{Hubert2013}

\section{Weekly Log}

\subsection{Week 1}

The goal of this week was to get familiar with the algorithm I will be working on. To do this I read through the main paper I will be working on (http://hal.inria.fr/hal-00668882). My first goal was to understand what Hermitian Normal Forms (HNFs) were. So I read up on them. HNFs fall into two categories, row and column, and this paper requires the use of both. The first issue I encountered was that the library I was using (specifically the Nemo Julia package) did not have an implementation for the column HNF. So I had to come up with my own implementation of it. While at first I looked at the algorithm (Cohen's algorithm from A Course in Computational Algebraic Number Theory), I realized that from the function could be implemented directly using the current `hnf` and `hnf\_with\_transform` function that already existed.

\section{Methodology}

In this section I will explain with an example the full methodology of the algorithm. The example I use is the the Schackenberg Model for a simple chemical reaction with limit cycle as shown in example 7.4 of \cite{Hubert2013}. The equation for this model is as follows.
\begin{align*}
    \diff{x}{t}
     & = a - kx + hx^2y \\
    \diff{y}{t}
     & = b - h x^2 y
\end{align*}

I explain the process to  work on this from now on.

\begin{enumerate}[label=Step \arabic*:]
    \item Form the function \(F\) with the Laurent polynomials of the differential equations.
\end{enumerate}

\newpage

\section{Guidelines for your thesis report}

\begin{itemize}
    \item Structure your thesis using sections, subsections, paragraphs as you like. We advise you to not use chapters (the thesis is a ``short'' document, it should not be more than 30 pages). However, you are free to use them if you think that will improve your work (so, at your own risk).
    \item Use bibtex to cite related work, e.g., \cite{DBLP:journals/x/Turing37}, adding your reference in the {main.bib} file and use the {$\backslash$cite} command.
    \item Change the value of the commands {$\backslash$thesistitle}, {$\backslash$authorname}, {$\backslash$supervisor}, {$\backslash$supervisorinstitution}, \\ {$\backslash$abstracttext} to set up the content of the title page. These commands are beginning of the {main.tex} file.
    \item The thesis should be \textbf{at most} 30 pages long. The official guideline is to have a thesis ``between 20 and 30 pages long''. However, these page limits are a \emph{guideline} and will not be enforced strictly.
          The main advise is to shoot for quality and not or length: we prefer reading fewer pages containing high quality, well explained, work rather than 30 pages of messy content. Your grade is not a function of the number of pages!
          Clearly, it's a red flag if your thesis length deviates a lot from the expected value of 20 pages (e.g., your thesis has 5 pages or it has 60 pages).
    \item The appendix is \textbf{not included} in the page count, while the title page, abstract page, table of content page, and all the bibliography instead are included. Again, the page limit is not enforced strictly, so this information should not be extremely important.
    \item Note: we will not evaluate your appendix that should not contain your thesis contribution but only provide additional details (e.g., some important code, additional tables or details for the experiments, \ldots).
    \item You \textbf{cannot} change the tex template (e.g., font size, margin size, \ldots). We fix a template so all the students work under the same condition. Get in touch if you see some bug in the template or some possible improvement (we may, or not, improve the template with your suggestion).
\end{itemize}

\newpage
\bibliographystyle{plain}
\bibliography{main}

\newpage
\appendix

\section{Appendix}
\label{sec:appendix}

\end{document}
